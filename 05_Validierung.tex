\chapter{Validierung}
\label{Validierung}
Hier werden Konzept und Implementierung in Hinblick auf die
Problemstellung validiert. Die eigentliche Validierung kann ganz
unterschiedlich erfolgen. Stehen quantitative Ziele im Vordergrund
(Performance, Latenz, Durchsatz, Skalierung, Kosten), kann die
Validierung aus Messungen und empirischen Untersuchungen
bestehen. Stehen technisch-funktionale Aspekte im Vordergrund (``ist
machbar''), kann die Validierung aus einer qualitativen Begutachtung
der eigenen Konzeption und Implementierung bestehen. Stehen soziale
oder ``menschliche'' Ziele im Raum, kann die Validierung Befragungen
oder Interviews enthalten. Kombinationen sind natürlich erlaubt.

Ziel ist eine gesamthafte, überzeugende und strukturierte Validierung
der eigenen Konzeption im Hinblick auf das Ziel. Ob das Ziel erreicht
wurde oder nicht, ist dabei im Grunde sekundär – auch negative
Resultate (``geht nicht'', ``ist zu langsam'') können sehr gute
Resultate sein. Wichtig ist das methodische Vorgehen zur Beantwortung
der Frage.

\section{Resultate}

\subsection{Das habe ich gemessen 1}
\label{Ergebnisse_1}

\subsubsection{Messungen unter Bedingung 1}

Hier stehen viele bunte Bilder. Also die Ergebnisse der Messungen/Rechnungen...

\subsubsection{Messungen unter Bedingung 2}

\subsection{Das habe ich berechnet 2}
\label{Ergebnisse_2}

\subsubsection{Rechenergebnisse mit Modell 1}

\subsubsection{Rechenergebnisse mit Modell 2}

\section{Interpretation}
\label{Interpretation}

Was bedeuten die ganzen Ergebnisse und was ist
das Fazit?\\ Während der Ergebnisteil neutral und objektiv die
Ergebnisse präsentieren soll, können hier verschiedene Ergebnisse
verglichen werden und Schlüsse gezogen werde sowie Bewertungen
abgegeben werden.
