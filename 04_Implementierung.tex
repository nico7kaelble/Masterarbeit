\chapter{Implementierung}
\label{Implementierung}

In der Konzeption wurden alle relevanten Entscheidungen getroffen, so
dass es in diesem Kapitel typischerweise nur noch um technische
Aspekte und Implementierungsdetails geht. Im Idealfall enthält dieses
Kapitel also keine ``Überraschungen'' auf konzeptioneller Ebene,
sondern beschreibt lediglich Besonderheiten der Implementierung, die
etwa jemand, der das System nachbauen oder nachvollziehen möchte,
wissen sollte. Zu beachten ist dabei, dass die Implementierung
grundsätzlich nur ein Mittel zur Beantwortung und Validierung der
Frage ist, und kein Selbstzweck. Sie dient – zumindest was die
Bearbeitung der Thesis betrifft – nur einem einzigen Ziel: der
Validierung (Kapitel \ref{Validierung}), ob der vorgestellte Entwurf
(Kapitel \ref{Ansatz}) die Problemstellung (Kapitel \ref{Einleitung}
tatsächlich erreicht.

Es kann (muss aber nicht zwingend) sinnvoll sein, das Kapitel
Implementierung analog zur Konzeption zu strukturieren. Hat man zum
Beispiel verschiedene Teile eines Gesamtsystems entworfen (in Kapitel
\ref{Ansatz}), kann man die Implementierung dieser Teile in analoger
Struktur (Kapitel \ref{Implementierung}) vorstellen.

\section{Umsetzung meines Verfahrens 1}
\label{MeinsImplementierung1}

Verfahren 1 habe ich komplett implementiert, getestet und produktiv
gesetzt, es läuft seit 10 Jahren fehlerfrei und macht insbesondere
meine Chefin total gluecklich.

\section{Umsetzung meines Verfahrens 2}
\label{Meins2Implementierung}

Verfahren 2 habe ich gar nicht umgesetzt, es bleibt komplettes
Papierdesign, aber es funktioniert bestimmt.
